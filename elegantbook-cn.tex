\documentclass[cn,11pt]{elegantbook}

\title{一点概括}
\subtitle{关于期末复习}

\author{罗大啸x赖泽磊x谢芳萍x黄伟煌x刘阳李子扬}
\institute{luodaxiao}
\date{\today}
\version{2.1}

\extrainfo{Victory won\rq t come to us unless we go to it. --- M. Moore}

\logo{logo1.png}
\cover{xmyz.jpg}



\begin{document}

\maketitle
\tableofcontents

% \thispagestyle{empty}

\mainmatter
\hypersetup{pageanchor=true}

\chapter{毛概重点}
\section{毛概课堂划题(已完善)}
\subsection{毛泽东思想的主要内容和活的灵魂?如何科学的评价毛泽东?}
{\large 
	
{\heiti  主要内容:}

\begin{itemize}
	\item 新民主主义革命理论
	\item 社会主义革命和社会主义建设理论
	\item 革命军队建设和军事战略理论
	\item 政策和策略的理论
	\item 思想政治工作和文化工作的理论
	\item 党的建设理论 
\end{itemize}

{\heiti 活的灵魂:}
\begin{itemize}
	\item 实事求是
	\item 群众路线
	\item 独立自主
\end{itemize}

{\heiti 如何科学地评价毛泽东:}毛泽东是伟大的马克思主义者、伟大的无产阶级革
命家、战略家和理论家。他为中国共产党和中国人民解放军的创立和发展,为中国各族人民解放事业的胜利,为中华人民共和国的缔造和社会主义事业的发展,建立了不可磨灭的功勋,为世界被压迫民族的解放和人类进步事业作出了重大贡献。

毛泽东的功绩是第一位的,错误是第二位的。他的错误是一个伟大的革命家、一个伟大的马克思主义者所犯的错误。}
\subsection{新民主主义的总路线和基本纲领。}
{\large 
{\heiti 新民主主义的总路线:}1948 年 4 月,毛泽东《在晋绥干部会议上的讲话》中,完整地表述了总路线的内容。即无产阶级领导的,人民大众的,反对帝国主义、封建主义和官僚资本主义的革命。\\
{\heiti 基本纲领:}\\
{\heiti 1.政治纲领:}新民主主义的政治纲领是:推翻帝国主义和封建主义的统治,建立一个无产阶级领导的、以工农联盟为基础的、各革命阶级联合专政的新民主主义的共和国。\\
{\heiti 2.经济纲领:}新民主主义的经济纲领是:没收封建地主阶级的土地归农民所有,没收官僚资产阶级的垄断资本归新民主主义的国家所有,保护农民工商业。\\
{\heiti 3.文化纲领:}新民主主义的政治和经济,必须要有与之相适应的新民主主义文化。新民主主义文化,就是无产阶级领导的人民大众的反帝反封建的文化,即民族的科学的大众的文化。

}
\subsection{新民主主义革命与旧民主主义革命的区别(29页),新民主主义革命与社会主义革命性质的不同?}
{\large 
{\heiti 新民主主义革命与旧民主主义革命的区别:}新民主主义革命与旧民主主义革命相比有新的内容和特点,集中表现在:{\heiti 1、}中国革命处于世界无产阶级社会主义革命的时代,是世界无产阶级社会主义革命的一部分;{\heiti 2、}革命的领导力量是中国无产阶级及其先锋队——中国共产党;{\heiti 3、}革命的指导思想是马克思列宁主义;{\heiti 4、}革命的前途是社会主义而不是资本主义。

{\heiti 新民主主义革命与社会主义革命性质的不同:}新民主主义革命仍属于资产阶级民主主义的范畴。它推翻帝国主义、封建主义和官僚资本主义的反动统治,在政治上争取和联合民族资产阶级去反对共同的敌人,在经济上保护民族工商业,容许有利国计民生的私人资本主义发展。它要建立的是无产阶级领导的各革命阶级的联合专政,而不是无产阶级专政。社会主义革命是无产阶级性质的革命,它所要实现的是消灭资本主义剥削制度和改造小生产的私有制。
}
\subsection{简述新民主主义道路形成的必然性(34-35页)}
{\large
	首先,中国必须走农村包围城市、武装夺取政权的道路,是由中国所处的时代特点和具体国情决定的。{\heiti 一方面},在半殖民地半封建的中国社会,内无民主制度而受封建主义的压迫,外无民族独立而受帝国主义的压迫。{\heiti 另一方面},近代中国是一个农业大国,农民占全国人口的绝大多数,是无产阶级可靠的同盟军和革命的主力军。
	\\有以下几个原因:
	
	第一,近代中国是多个帝国主义间接统治的经济落后的半殖民地国家,社会政治经济发展极端不平衡,四分五裂、军阀割据,存在不少的统治薄弱环节,为党在农村开展革命斗争、建设革命根据地{\heiti 提供了缝隙和可能};
	
	第二,近代中国的广大农村深受反动 统治阶级的多重压迫和剥削,人民革命愿望强烈,加之经历过大革命的洗礼,{\heiti 革命的群众基础好};
	
	第三,全国革命形势的继续向前发展,为在农村建设革命根据地提供了{\heiti 客观条件};
	
	第四,相当力量正式红军的存在,为农村革命的创立、巩固和发展{\heiti 提供了坚强后盾};
	
	第五,党的领导的有力量及其政策的不错误,为农村革命根据地建设和发展{\heiti 提供了重要的主观条件}。
	
	
 }
\subsection{党在过渡时期的总路线及其理论依据,社会主义确立的重大意义(61页)}
{\large 

1953年6月,毛泽东在中央政治局会议上正式提出过渡时期的总路线和总任务,同年12月形成关于总路线的完整表述:“从中华人民共和国成立,到社会主义改造基本完成,这是一个过渡时期。党在这个过渡时期的总路线和总任务,是要在一个相当长的时期内,逐步实现国家的社会主义工业化,并逐步实现国家对农业、对手工业和对资本主义工商业的社会主义改造。”

 社会主义基本制度的确立是中国历史上最深刻最伟大的社会变革,为当代中国一切发展进步奠定了制度基础,也为中国特色社会主义制度的创新和发展提供了重要前提。
 
 1.社会主义基本制度的确立,极大地提高了工人阶级和广大劳动人民的积极性、创造性,极大地促进了我国社会生产力的发展。
 
 2.社会主义基本制度的确立,使广大劳动人民真正成为国家的主人。
 
 3.社会主义基本制度的确立,使占世界1/4的东方大国进入了社会主义社会,增强了社会主义的力量,对维护世界和平产生了积极影响。
 
 社会主义基本制度的确立,是以毛泽东为主要代表的中国共产党人对一个脱胎于半殖民地半封建的东方大国如何进行社会主义革命问题的系统回答和正确解决,是马克思主义关于社会主义革命理论在中国的正确运用和创造性发展的结果。它不仅再次证明了马克思主义的真理性,而且以其独创性的理论原则和经验总结丰富和发展了科学社会主义理论。
 
}
\subsection{社会主义初步探索的意义和经验教训(77-78页)}
{\large 
{\heiti 意义:}1、巩固和发展了我国社会主义制度。
2、为开创中国特色社会主义提供了宝贵经验、理论准备、物质基础。3、丰富了科学社会主义的理论和实践。

{\heiti 经验教训:}\\
1、必须把马克思主义和中国实际相结合,探索符合中国特点的社会主义建设道路。\\
2、必须正确认识社会主义社会的主要矛盾和根本任务,集中力量发展生产力。\\
3、必须从实际出发进行社会主义建设,建设规模和速度要和国力相适应,不能急于求成。\\
4、必须发展社会主义民主,健全社会主义法制。\\
5、必须坚持放的民主集中制和集体领导制度,加强执政党建设。\\
6、必须坚持对外开放,借鉴和吸收人类成果建设社会主义,不能关起门来搞建设。


}
\subsection{邓小平理论的基本问题和主要内容(94-97..页)}
{\large 
{\heiti 问题:}什么是社会主义、怎样建设社会主义?

{\heiti 主要内容:}\\1、解放思想、实事求是的思想路线\\
2、社会主义初级阶段理论\\
3、党的基本路线\\
4、社会主义根本任务的理论\\
5、“三步走”战略\\
6、改革开放理论\\
7、社会主义市场经济理论\\
8、“两手抓,两手都要硬”\\
9、“一国两制”\\
10、中国问题的关键在于党\\

}
\subsection{“三个代表”重要思想的核心观点和主要内容(123页)}
{\large 
{\heiti 核心观点:}1.始终代表中国先进生产力的发展要求。2.始终代表中国先进文化的前进方向。3.始终代表中国最广大人民的根本利益。

{\heiti 主要内容:}\\1.发展是党执政兴国的第一要务。党要承担起推动中国社会进步的历史责任,必须始终紧紧抓住发展这个执政兴国的第一要务。\\2.建立社会主义市场经济体制。在社会主义条件下发展市场经济,实现了改革开放新的历史性突破,打开了我国经济、政治和文化发展的崭新局面。\\3.全面建成小康社会。全面建成小康社会的奋斗目标,是立足于我国的基本国情提出的,是实现现代化建设第三步战略目标必经的承上启下的发展阶段,也是完善社会主义市场经济体制和扩大对外开放的关键阶段。\\4.建设社会主义政治文明。发展社会主义民族政治,建设社会主义政治文明,是社会主义现代化的重要目标。\\5.推进党的建设新的伟大工程。一定要从新的实际出发。以改革的精神研究和解决党的建设面临的重大理论和现实问题,使党始终保持先进性和纯洁性,充满创造力、凝聚力和战斗力,推进党的建设新的伟大工程。
}
\subsection{科学发展观的科学内涵和主要内容(152-167页)}
{\large 

{\heiti 科学发展观的科学内涵:}科学发展观,第一要义是发展,核心立场是以人为本,基本要求是全面协调可持续,根本方法是统筹兼顾。

{\heiti 科学发展观的主要内容:}\\
1.加快转变经济发展方式\\
2.发展社会主义民主政治\\
3.推进社会主义文化强国建设\\
4.构建社会主义和谐社会\\
5.推进社会主义文明建设\\
6.全面提高党的建设科学文化水平


}
\subsection{新时代的内涵和意义(181-182)}
{\large 
{\heiti 内涵:}

第一,这个新时代是承前启后、继往开来,在新的历史条件下继续夺取中国特色社会主义伟大胜利的时代。

第二,这个新时代是决胜全面建成小康社会、进而全面建设社会主义现代化强国的时代。

第三,这个新时代是全国各族人民团结奋斗、不断创造美好生活、逐步实现全体人民共同富裕的时代;

第四,这个新时代是全体中华儿女勠力同心、奋力实现中华民族伟大复兴中国梦的时代。

第五,这个新时代是我国日益走近世界舞台中央、不断为人类作出更大贡献的时代。

{\heiti 意义:}

第一,从中华民族复兴的历史进程看,中国特色社会主义进入新时代,意味着近代以来久经磨难的中华民族迎来了从站起来、富起来到强起来的伟大飞跃,迎来了实现中华民族伟大复兴的光明前景。

第二,从科学社会主义发展进程看,中国特色社会主义进入新时代,意味着科学社会主义在 21 世纪的中国焕发出强大生机活力,在世界上高高举起了中国特色社会主义伟大旗帜。

第三,从人类文明进程看,中国特色社会主义进入新时代,意味着中国特色社会主义道路、理论、制度、文化不断发展,拓展了发展中国家走向现代化的途径,给世界上那些既希望加快发展又希望保持自身独立性的国家和民族提供了全新选择,为解决人类问题贡献了中国智慧和中国方案
}
\subsection{习近平新时代中国特色社会主义思想的核心要义和丰富内涵(183页)}
{\large 
{\heiti 核心要义:}坚持和发展中国特色社会主义,是改革开放以来我们党全部理论和实践的鲜明主题,也是习近平新时代中国特色社会主义思想的核心要义。

{\heiti 丰富内涵:}

第一,明确坚持和发展中国特色社会主义,总任务是实现社会主义现代化和中华民族伟大复兴,在全面建成小康社会的基础上,分两步走在本世纪中叶建成富强民主文明和谐美丽的社会主义现代化强国。

第二,明确新时代我国社会主要矛盾是人民日益增长的美好生活需要和不平衡不充分的发展之间的矛盾,必须坚持以人民为中心的发展思想,不断促进人的全面发展、全体人民共同富裕。

第三,明确中国特色社会主义事业总体布局是“五位一体”、战略布局是“四个全面”,强调坚定道路自信、理论自信、制度自信、文化自信。

第四,明确全面深化改革总目标是完善和发展中国特色社会主义制度、推进国家治理体系和治理能力现代化。

第五,明确全面推进依法治国总目标是建设中国特色社会主义法治体系、建设社会主义法治国家。

第六,明确党在新时代的强军目标是建设一支听党指挥、能打胜仗、作风优良的人民军队,把人民军队建设成为世界一流军队。

第七,明确中国特色大国外交要推动构建新型国际关系,推动构建人类命运共同体。

第八,明确中国特色社会主义最本质的特征是中国共产党领导,中国特色社会主义制度的最大优势是中国共产党领导,党是最高政治领导力量,提出新时代党的建设总要求,突出政治建设在党的建设中的重要地位。
}
\subsection{坚持和发展中国特色社会主义的基本方略(185页)}
\begin{itemize}
	\item 坚持党对一切工作的领导。
	\item 坚持以人民为中心。
	\item 坚持全面深化改革
	\item 坚持新发展理念。
	\item 坚持人民当家作主
	\item 坚持全面依法治国
	\item 坚持社会主义核心价值体系
	\item 坚持在发展中保障和改善民生
	\item 坚持人与自然和谐共生
	\item 坚持总体国家安全观
	\item 坚持党对人民军队的绝对领导
	\item 坚持“一国两制”和推进祖国统一
	\item 坚持推动构建人类命运共同体
	\item 坚持全面从严治党
\end{itemize}
{\large }
\subsection{社会主义核心价值观和价值体系的关系(226页),如何培育和践行核心价值观(227页)}
{\large 
{\heiti 关系:}社会主义核心价值观是在社会主义核心价值体系基础上提炼出来的。社会主义核心价值观是社会主义核心价值体系的内核凝练和集中表达。{\heiti 一方面},二者方向一致,都体现了社会主义意识形态的本质要求,体现了社会主义制度在思想和精神层面的质的规定性,凝结着社会主义先进文化的精髓,是中国特色社会主义道路、理论、制度和文化,是实现中华民族伟大复兴的中国梦的价值引领。{\heiti 另一方面},二者各有侧重,相比于社会主义核心价值观体系,社会主义核心价值观更加突出核心要素、更加注重凝练表达、更加强化实践导向。

{\heiti 如何做?}\\
1.培育和践行社会主义核心价值观,要把社会主义核心价值观融入社会生活各个方面。\\2.培育和践行社会主义核心价值观,要坚持全民行动、干部带头,从家庭做起、从娃娃抓起。\\3.培育和践行社会主义核心价值观,必须立足中华优秀传统文化和革命文化。\\4.培育和践行社会主义核心价值观,还必须发扬中国人民在长期奋斗中培育、继承、发展起来的伟大民族精神。


}
\subsection{建设现代经济体系的基本要求和主要任务}
{\large 
{\heiti 基本要求:}贯彻新发展理念:\\
1.创新是引领发展的第一动力。\\
2.协调是持续健康发展的内在要求。\\
3.绿色是永续发展的必要条件。\\
4.开放是国家繁荣发展的必由之路。\\
5.共享是中国特色社会主义的本质要求。\\
6.创新、协调、绿色开放、共享的新发展理念,相互贯通、相互促进,是具有内在联系的集合体。

{\heiti 主要任务:}\\
1.要建设创新引领、协同发展的产业体系,实现实体经济、科技创新、现代金融、人力资源协同发展,使科技创新在实体经济发展中的贡献份额不断提高,人力资源支撑实体经济发展的作用不断优化。\\
2.要建设统一开放、竞争有序的市场体系,实现市场准入畅通、市场开放有序、市场竞争充分\\
3.要建设体现效率、促进公平的收入分配体系,实现收入分配合理、社会公平正义、全体人民共同富裕,推进基本公共服务均等化,逐渐缩小收入分配差距。\\
4.要建设彰显优势、协调联动的城乡区域发展体系,实现区域良性互动、城乡融合发展、塑造区域协调发展新格局。\\
5.要建设资源节约、环境友好的绿色发展体系。\\
6.要建设多元平衡安全高效的全面开放的体系,发展更高层次开放型经济\\
7.要建设充分发挥市场作用、更好发挥政府作用的经济体制,实现市场机制有效、微观主体有活力、宏观调控有度。
}
\subsection{如何共商共建人类命运共同体}
{\large
构建人类命运共同体思想,是一个科学完整、内涵丰富、意义深远的思想体系,其核心就是“建设持久和平、普遍安全、共同繁荣、开放包容、清洁美丽的世界”。

第一,坚持和平发展道路,推动建设新型国际关系。

第二,不断完善外交布局,积极发展全球伙伴关系。

第三,深度参与全球治理,积极引导国际秩序变革
方向。

第四,推动国际社会从伙伴关系、安全格局、经济
发展、文明交流、生态建设等方面为建立人类命运共同
体作出努力。

}

\subsection{我们遇到的严重生态问题和环境问题是什么,面对这些环境问题应当树立什么样的生态文明理念,如何实现这些生态文明理念
}
{\large 
{\heiti 问题:}到了近代,随着工业化的到来,和世界许多国家一样,我们也经历了一个向自然界进军、改造自然、征服自然的过程,在快速形成现代化发展物质基础的同时,也给自然生态系统带来了很大的破坏,出现森林消失、土地沙化、湿地退化、水土流失、干旱缺水等严重生态问题和水、土、空气遭到污染等严重环境问题。

{\heiti 如何做?}\\1.尊重自然,是人与自然相处时应秉持的首要态度,要求人对自然怀有敬畏之心、感恩之情、报恩之意,尊重自然界的创造和存在,绝不能凌驾于自然之上,只有尊重自然才是人与自然相处的科学态度。\\2.顺应自然,是人与自然相处时应遵循的基本原则,要求人顺应自然的客观规律,按自然规律办事。\\3.保护自然,是人与自然相处时应承担的重要责任,要求人发挥主观能动性,在向自然界索取生存发展之需的同时,呵护自然,回报自然,保护自然界的生态系统,对自然界不能只索取不讲投入、只讲利用不讲建设。
}
\subsection{全面深化改革的总目标和主要内容(6个紧紧围绕)}
{\large 
{\heiti 总目标:}完善和发展中国特色社会主义制度,推进国家治理体系和治理能力现代化。

{\heiti 主要任务:}\\
1.强调要紧紧围绕使市场资源配置中起决定性作用和更好发挥政府作用深化经济体制改革;\\2.紧紧围绕坚持坚持党的领导、人民当家作主、依法治国有机统一深化政治体制改革;\\3.紧紧围绕建设社会主义核心价值体系社会主义文化强国深化文化体制改革;\\紧紧围绕更好保障和改善民生、促进社会公平正义深化社会体制改革;\\4.紧紧围绕建设美丽中国深化生态文明体制改革;\\5.紧紧围绕提高科学执政、民主执政、依法执政水平深化党的建设制度改革。


}
\subsection{为什么要坚持党对军队的绝对领导,其基本内容是?}
{\large 

党对军队的绝对领导是中国特色社会主义的本质特征,是党和国家的重要政治优势。推进强军事业,必须毫不动摇坚持党对军队的绝对领导,确保任命军队永远听党话、跟党走。党的领导是人民军队战无不胜的根本保证。人民军队从诞生之日起,就始终在党的绝对领导下行动和战斗。。历史告诉我们,党指挥枪是保证人民军队本质和宗旨的根本保障,这是我们党在血与火的斗争中颠扑不破的真理。有了中国共产党,有了中国共产党的领导,人民军队前进就有方向。

{\heiti 基本内容:}\\一、军队必须完全无条件的置于中国共产党的领导下\\二、在思想上政治上行动上始终与党中央、中央军委保持一致
\\三、坚决维护党中央、中央军委权威,\\四、在任何时候任何情况下都坚决听从党中央、中央军委指挥}
\subsection{新时代为什么要从严治党?新时代党的建设基本要求是什么?}
{\large 
{\heiti 新时代为什么要从严治党:}进入新时代,我们党面临的执政环境是复杂的,影响党的先进性、弱化党的纯洁性的因素也是复杂的,党内思想不纯、组织不纯、作风不纯等突出问题尚未得到根本解决。党面临的执政考验、改革开放考验、市场经济考验、外部环境考验具有长期性和复杂性,党面临的精神懈怠危机、能力不足危险、脱离群众危险、消极腐败危险具有尖锐性和严峻性。推进党的建设新的伟大工程要一以贯之

{\heiti 基本要求:}坚持加强党的全面领导,坚持党要管党、全面从严治党,以加强党的长期执政能力建设、先进性和纯洁性建设为主线,以党的政治建设为统领,以坚持理想信念宗旨为根基,以调动全党积极性、主动性创造性为着力点,全面推动党的政治建设、思想建设、作风建设、组织建设、纪律建设,把制度建设贯穿其中,深入推进番腐败斗争,不断提高建设党的建设质量,把党建设成始终在时代前列、人民衷心拥护、勇于自我革命、经得起各种风浪考验、朝气蓬勃的马克思主义执政党。
}
\subsection{如何理解党的领导是中国特色最本质的特征(298页)}
{\large 
这一论断符合科学社会主义的基本原则,反映中国特色社会主义的历史经验,适应新时代历史使命的实践要求。

第一,这是由科学社会主义的理论逻辑所决定的。坚持无产阶级政党的领导是无产阶级革命和社会主义建设取得胜利的根本保证。

第二,这是由中国特色社会主义产生于发展的历史逻辑所决定的。中国特色社会主义不是从天上掉下来的,而是在改革开放40年的伟大实践中得来的,是党和人民历经千辛万苦、付出各种代价取得的宝贵成果。

第三,这是由中国特色社会主义迈向新征程的实践逻辑所决定的。实现中华民族的伟大复兴,关键在党。
}
\subsection{结合中国革命历史进程谈一谈马克思主义中国华如何能从中国站起来,富起来强起来的历史性飞跃?}
{\large 
第一次历史性飞跃发生在新民主主义革命时期,形成了毛泽东思想。第二次历史性飞跃发生在社会主义进入改革开放的新时期,形成了包括邓小平理论,“三个代表“重要思想,科学发展观,习近平新时代中国特色社会主义思想在内的中国特色社会主义理论体系。

毛泽东思想是在革命和建设的长期实践中,以毛泽东为主要代表的中国共产党人,根据马克思列宁主义基本原理,形成的适合中国情况的科学指导思想,完成新民主主义革命,建立了中国人民共和国,近代以来久经磨难的中华民族从此站起来了。
消灭一切剥削制度,确立社会主义基本制度,推进社会主义建设,完成了中华民族有史以来最为广泛而深刻的社会变革,为当代中国一切发展进步奠定了根本政治前提和制度基础。

以邓小平为代表的中国共产党人,重新确立了实事求是的思想路线,在邓小平理论的指导下,20世纪的中国又一次发生了天翻地覆的变化,开启了中华民族“富起来”的新征程。
以江泽民为主要代表的中国共产党人,形成了“三个代表”重要思想。成功把中国特色社会主义推向21世纪。

以胡锦涛为代表的中国共产党人,形成了以人为本,全面协调可持续发展的科学发展观,在新的历史起点上坚持和发展了中国特色社会主义。

党的十八大以来,以习近平为代表的中国共产党人,创立了习近平新时代特色社会主义思想。在习近平新时代特色社会主义思想指导下,中国共产党领导全国各族人民,统揽伟大斗争,伟大工程,伟大事业,伟大梦想,推动中国特色社会主义进入了新时代,推动中华民族迎来了从站起来,富起来到强起来的伟大飞跃。

}
\chapter{关于模电的一点总结}
考后总结:考题跟肖老师划的重点完全不同。
\section{关于OCL电路}
电路图可参考模电书411页-图8.2.1 消除交越失真的OCL电路
\subsection{求解OCL电路负载上可能获得的最大功率和效率}
{\large 最大功率:}
$$P_{om}=\dfrac{(V_{cc}-|U_{CES}|)^{2}}{2R_{L}}$$

{\large 效率:}
$$\eta=\dfrac{\pi}{4}\dfrac{V_{CC}-|U_{CES}|}{v_{cc}}$$
\subsection{当已知输出电压$U_{om}$时最大输出功率}
$$P_{om}=\dfrac{U_{om}^{2}}{R_{L}}$$
\subsection{晶体管最大功耗}
$$P_{Tmax}=\dfrac{V_{cc}^{2}}{\pi^{2}R_{L}}$$
\section{关于波形发生电路}
\subsection{正弦波振荡电路}
\subsubsection{电路的组成:放大电路、选频网络、正反馈网络、稳幅环节}
{\large 
	
	在不少实用电路中,常将选频网络和正反馈网络“合二为一”;而且,对于分立元件放大电路,也不再另加稳幅环节,而依靠晶体管特性的非线性来达到稳幅作用。
}
\subsubsection{判断电路是否可能产生正弦波}
\begin{itemize}
	\item 观察电路是否包含了放大电路、选频网络、正反馈网络和稳幅环节四个组成部分;
	
	\item 判断电路是否能正常工作,即静态工作点等;
	
	\item 瞬时极性法判断相位条件,即输入量和反馈量同相(正反馈);
	
	\item 判断幅值条件$|AF|>1$
\end{itemize}
\subsection{RC正弦波振荡电路}
{\large 
	此电路是利用RC串并联选频网络的。\\	
	要求:电压放大倍数$A_{u}=3$,即当$f=f_{0}$时输入输出电压同相且放大倍数的值等于3。考虑到起振条件,我们一般使放大倍数略大于3。
通常由RC选频网络和同相比例运算电路构成RC正弦波振荡电路。电路图见书本345页图7.1.6

需掌握如何计算振荡频率:$f=\dfrac{1}{2\pi RC}$

掌握如何计算同相比例运算电路的放大倍数从而使达到$A_{u}>3$,$$A_{u}=(1+\dfrac{R_{f}}{R})$$

其中$R_{f}$为反馈电阻,R为方向端所接电阻。
}
\subsection{LC正弦波振荡电路}
{\large LC正弦波振荡电路分为三种,它们都有LC并联作为选频网络,但是他们的反馈方式不同,分别为变压器反馈、电容反馈和电感反馈。\\在这里主要是要掌握判断电路是否能够产生正弦波。比如电路的静态工作点、交流通路有无开路短路,还有就是相位条件是否满足。(输入量和反馈量同相,即正反馈)}
\section{电压比较器}
\begin{itemize}
	\item 单限比较器 \item 滞回比较器 \item 窗口比较器
\end{itemize}
这里一般就是要求解阈值电压,就是用三要素法来分析电路。然后这三种电压比较器的电路图和输出电压图要了解。
\section{矩形波发生电路}
{\large 
	矩形波发生电路是由滞回比较器和RC延时组成,经典电路图见模电书(),这个电路一般需要记住。且需要掌握其周期的计算方法。\\
	这里给出方波发生电路的周期计算方法:
	$$T=2R_{3}Cln(1+\dfrac{2R_{1}}{R_{2}})$$
	\\其占空比$q=\dfrac{1}{2}$
	\\占空比可调的矩形波发生电路周期计算方法:
	$$T=(R_{w}+2R_{3})ln(1+\dfrac{2R_{1}}{R_{2}})$$
	\\其占空比为$q=\dfrac{R_{3}+R_{w1}}{2R_{3}+R_{w}}$
	 \\ 
	 ~\
	 ~\
	\emph{注:此处公式均为书本上经典电路图的求解方法。}
}
\section{三角波发生电路}
三角波发生电路的周期计算:$$T=\dfrac{4R_{1}R_{3}C}{R_{4}}$$
\section{运算放大电路}
电路图见模电书第六章。
~\
\begin{itemize}
	\item 反相比例运算电路
	$$U_{o}=-\dfrac{R_{f}}{R}U_{i}$$
	\item 同相比例运算电路
	$$U_{o}=(1+\dfrac{R_{f}}{R}U_{i})$$
	\item 电压跟随器
	$$U_{o}=U_{i}$$
	\item 求和运算电路(同相、反相)
	$$U_{o}=R_{f}(\dfrac{U_{i1}}{R_{1}}+\dfrac{U_{i2}}{R_{2}})$$
	$$U_{o}=-R_{f}(\dfrac{U_{i1}}{R_{1}}+\dfrac{U_{i2}}{R_{2}})$$
	\item 加减运算电路
	$$U_{o}=R_{f}(\dfrac{U_{i1}}{R_{1}}-\dfrac{U_{i2}}{R_{2}})$$
	\item 积分运算电路
	$$U_{o}=-\dfrac{1}{RC}\int_{t_{0}}^{t_{1}}U_{i}dt + U_{o}(t_{0})$$
	\item 微分运算电路
	$$U_{o}=-RC\dfrac{dU_{i}}{dt}$$
\end{itemize}
\section{关于反馈}
\subsection{一些判断}
有无反馈?\ 反馈极性?\ 交流反馈直流反馈?
\subsection{反馈的组态}
反馈的判断技巧:若反馈元件直接连接在输入端时,为并联反馈,反之为串联;若反馈元件直接接在输出端或分压端,为电压反馈,反之为电流反馈。\\
{\heiti 注意}:找对反馈元件,在判断电压电流反馈时观察其\ {\large 两端}\ 的接法.
\begin{itemize}
	\item 电压串联反馈\item 电压并联反馈 \item 电流串联反馈 \item 电流并联反馈
\end{itemize}
\chapter{半导体器件物理(...)}

{\zihao {4} 作者:luodaxiao\\本章仅限作者复习使用}
\section{半导体基础}
\subsection*{一、简答题}
\subsubsection*{1.简述能带的性质。}
(1)、由于。。。\\
因此能带的能量谱值具有周期性,即$$E_{n}(K)=E_{n+n_{k}}(K)$$
说明:能带在空间中的周期性重复并不能产生新的状态,不能简并化,因此将K限制在第一布里渊区。

(2)、能量谱值E具有对K的宏观对称性,即$$E_{n}(K)=E_{n}(aK)$$
(3)、E是k的偶函数,即$$E_{n}(K)=E_{n}(-k)$$
(4)、能带在空间中的排布可能相互堆叠也可能存在间隙,各允带之间的间隙称为禁带或带隙。
\subsubsection*{2.简述玻恩-冯·卡门的周期性边界条件。}
假设在空间中有一有限晶体在一个无限晶体内部,这个无限晶体是由有限晶体周期性重复堆叠而成,有:\\
(1)、由于有限晶体在无限晶体内部,电子在有限晶体界面周围的情况与内部相同,因此,电子势场的周期性不至于被破坏;\\
(2)、假设的无限晶体是有限晶体的周期性重复,规定电子的运动情况是有限晶体在空间中周期性重复,因此只需要考虑有限晶体就够了。
\subsubsection*{3.根据玻恩-冯·卡门的周期性边界条件,简述波矢量 k 的性质。}
性质:\\
(1)、k代表电子在空间中的运动情况,每个k代表电子的一种运动情况;\\
(2)、将k限制在第一布里渊区;\\
(3)、在第一布里渊区k取分立值;\\
(4)、每个k所占体积为$(2\pi)^{3}/V$\\
(5)、k的状态密度为$V/(2\pi)^{3}$\\
(6)、每个倒原胞中,k代表的点数等于总原胞数N
\subsubsection*{4.什么是简并半导体?什么是 n 型简并半导体?什么是 p 型简并半导体?分别画出他们的能带图。}
简并半导体:在重参杂条件下,费米能级$E_{F}$接近或进入能带的现象称为载流子的简并化,发生载流子的简并化的半导体称为简并半导体。

n型简并半导体:导带中电子浓度大于状态密度$N_{v}$,费米能级位于导内部带的半导体;
p型简并半导体:价带中空穴浓度大于状态密度$E_{c}$时,费米能级位于价带内部的半导体。
\subsubsection*{5.电场作用下,为什么满带中的电子不能导电?为什么不满带中的电子可以导电?}
在电场的作用下,满带中电子状态在布里渊区的排布是对称的,也就是所电场的作用不能给以电子净动量,因此即使有电场的作用,满带中电子仍不导电;

不满带中电子在电场的作用下,电子状态在布里渊区的排布是不对称的,另外,由于晶格振动和杂质等对电子的辐射使电子状态有趋于热平衡的趋势,这两种作用使电子在布里渊区有一个相对稳定的状态。此时,占据与电场方向同向的电子数量较少,与电场方向反向的电子数量较多,电子产生的电流不能抵消,总电流不为零,因此不满带中的电子可以导电。
\subsubsection*{6.根据电子填充能带的情况,说明导体、半导体、绝缘体的导电机理。并画出相应的能带结构。}
(1)、对于导体,其电子占据的最高能带是不满带,且电子的密度很大,与原子的密度相当,因此具有良好的导电性;

(2)、对于半导体和绝缘体,在绝对零度时它们电子占据的最高能带是满带,且与之相邻的上一级能带是空带,满带与空带之间被禁带隔开,由于不存在不满带,因此它们是不导电的。

对于绝缘体来说,由于禁带宽度很宽,在温度升高的情况下,电子也难以从满带激发到价带中去,因此绝缘体是不导电的;

(3)、由于半导体的禁带宽度比绝缘体的窄,在温度升高时,满带中的电子可以通过热激发到达价带中,这时满带由于失去电子变为不满带,可以导电,空带由于的到电子变为不满带,可以导电,因此而具有导电性。随着温度的升高,热激发的电子数急剧增多,这就是随着温度的升高半导体的电导率逐渐升高的根本原因。

\subsubsection*{7.什么是直接复合?分别从载流子运动和能带的角度解释直接复合。}
直接复合:电子从导带中直接跃迁到价带的空状态,使电子和空穴成对消失;

载流子运动角度:电子和空穴在运动中总有一定概率直接相遇而复合;

能带角度:电子在导带和价带之间跃迁而引起非平衡载流子复合的过程就是直接复合。
\subsubsection*{8.什么是间接复合?描述间接复合的四个过程。}
间接复合:通过复合中心复合;

间接复合分为四个过程:\\
过程甲表示电子被复合中心俘获的过程,过程乙是过程甲的逆过程,它表示复合中心上的电子跃迁到导带的空状态,过程丙是空穴被复合中心俘获的过程,它表示空穴从复合中心跃迁到价带或者电子从价带跃迁到复合中心的空状态。
\subsection*{二、计算题}
{\zihao {-4} 本章的计算题主要要求掌握费米能级、载流子浓度(n、p)、准费米能级的计算,对于这几个的计算掌握了就不用慌了。
	
费米能级的计算:$E_{F}-E_{i}=KTln\dfrac{n}{n_{i}}$\ \ ,准费米能级的计算需先计算$\delta$n,然后分别算电子和空穴的能级:$E_{Fn}-E_{i}$和$E_{i}-E_{Fp}$}
\subsubsection*{1.一块硅片掺磷浓度为 $10^{15}cm^{-3}$。求室温下(300k)的载流子浓度和费米能级。(硅的$n_{i}=1.5*10^{10}cm^{-3}$,$V_{T}=0.026V$)}

\section{p-n\ junction}
\subsection*{一、简答题}
\subsubsection*{1、简述 pn 结空间电荷区的形成过程。}

当p型半导体与n型半导体接触时,P型半导体和n型半导体的交界面处形成一个pn结(冶金结)。在冶金结上,电子和空穴的浓度有很大的浓度梯度。在pn结中由于载流子浓度不同,n区的多子电子向p区扩散,p区的多子空穴向n区扩散。在无外加电路的情况下,这种扩散不会无限继续下去。

由于n区电子向p区扩散,留下带正电的施主离子,同样由于p区空穴向n区扩散,使带负电的
受主离子留下。在Pn结中,由于n区和p区分别带净正电荷和净负电荷,在pn结附近感生出一个内建电场,电场方向由n指向p。

由于内建电场的存在,空间电荷区中的电子和空穴都被扫出,使空间电荷区中无任何可自由移动的离子,因此也被叫做耗尽层。
\subsubsection*{2、什么是pn结的单向导电性(又称整流特性)?简述正偏、反偏时,pn 结的能级、内建电势差、势垒高度及空间电荷区宽度的变化情况。}
pn结加正偏电压时,势垒高度降低,有助于载流子通过空间电荷区,产生大的电路,形成阻抗低的电流通路;加反向偏压时,势垒高度升高,升高的势垒阻碍载流子通过空间电荷区,电流很小,pn结阻抗很高。以上分析说明pn结具有单向导电性,也被称为整流特性。

加正偏电压V时,n区的费米能级相对于p区的费米能级升高qV,其余n区相应能级也升高qV.

内建电势差下降为:fai0-v

内建电场下降为:q(fai0-V)

势垒高度下降,空间电荷区减小

加反偏电压$V_{r}$时:n区的费米能级相对于p区的费米能级降低qV,其余n区相应能级也降低qV.

内建电势差升高至:fai0+vr

内建电场升高为:q(fai0+V)

势垒高度升高,空间电荷区变宽。
\subsubsection*{3、什么是正偏复合电流?正偏复合电流的性质有哪些?}

正偏电压注入载流子通过空间电荷区,使空间电荷区的载流子浓度有可能大于平衡浓度,使np>$(n_{i})^{2}$,造成空间电荷区非平衡载流子的复合,这个复合电流就是正偏复合电流。

性质:\\
1.外加偏压越小,复合电流越显著。随着电压的逐渐增大,扩散电流逐渐占主导。\\
2.pn结材料的禁带宽度越大,复合电流越大。\\
3.轻掺杂的一端浓度越大,产生的复合中心越多,复合电流越大。
\subsubsection*{4、什么是隧道电流?隧道电流是如何产生的?}
pn结的两端都为重参杂的情况下,由于量子力学的隧穿效应,有些载流子可以隧穿势垒形成额外的电流,这就是隧道电流。

隧道电流在以下情况下产生:\\
1.费米能级进入能带,即费米能级位于导带或价带内部。\\
2.空间电荷区很窄,载流子有更大概率穿透。\\
3.在相同的能量级别时,能带的一侧有电子而另一侧为空的状态。

在重参杂情况下,即pn结简并化,满足条件一和二。在外加偏压的情况下,满足条件三。
\subsubsection*{5、什么是 pn 结的扩散电容?扩散电容具有哪些性质?}
外加正偏电压注入并存储在扩散区的少数载流子电荷由于外加电压的变化而产生的电容。

性质:\\
1.外加正偏电压时才有扩散电容,电压越大,扩散电容越显著。外加反偏电压时,存储电荷几乎不存在,扩散电容可以忽略。\\
2.工作电流越大扩散电容越大。\\
3.当$w\tau_{d}>>1\ w\tau_{a}>>1$工作在高频情况下,Cd很小,存储电荷跟不上外加偏压的变化可以忽略。在$w\tau_{d}<<1\ w\tau_{a}<<1$时,扩散电容特别重要。\\
4.缩短少子寿命(惨金)可以减小扩散电容。
\subsubsection*{6、什么是 pn 结二极管的开关特性?什么是电荷存储?什么是 pn 结反向瞬变?反向瞬变的产生原因是什么?}
当外加正向偏压时,pn结可以流过的电流很大,当外加反向偏压时,pn结允许流过的电流很小,我们把加正偏的 pn 结二极管称为开状态,把反偏的 pn 结二极管称为关状态。可见 pn 结二极管具有开关特性。

电荷存储:外加正偏压将载流子注入并存储在pn结二极管中,叫电荷存储。

pn结的反向瞬变:pn结两端的电压由正偏突然变为反偏,二极管中的存储电荷来不及消除,叫做反向瞬变。

原因:由于电荷存储效应。
\subsubsection*{7、pn 结结击穿的机制有哪些?详细解释这几种击穿机制。}

机制:齐纳击穿,雪崩击穿\\
齐纳击穿:在高电场时,由于电子的隧穿效应引起的击穿。在反向偏压作用下,由于p区和n区靠的很近,P区价带中的电子能够到达n区的导带中,例如硅pn结在低于4V时的击穿.\\
雪崩击穿:在高电压下,由于载流子的雪崩效应而产生的击穿。例如硅在高于6V时的击穿,雪崩效应:当载流子通过空间电荷区。由于电场的作用,载流子获得能量,当能量积累到一定值时,载流子会撞击耗尽层中的电子而形成新的电子空穴对,新的电子和空穴又会撞击其他电子。这就是雪崩效应。


\section{双极结型晶体管}
\subsection*{一、简答题}
\subsubsection*{1、以共基极接法的晶体管工作在正向有源模式为例,解释双极结型晶体管放大作用的基本原理。}

以共基接法为例。当基极既工作在输入回路又工作在输出回路,这种接法叫做BJT的共基极接法。由于发射结正偏,电子从发射区注入基区,空穴从基区注入发射区,使基区电子浓度很大,发射区空穴浓度很大,载流子的浓度取决于外加偏压的大小和掺杂浓度。当基区很薄时,只有少部分的电子被复合,其余绝大多数的电子都能到达集电区的边界空间电荷区。对于集电区加反偏电压,使电子和空穴都扫入空间电荷区,形成集电区电流,这个电流远大于集电结反偏电流。由以上分析可知,输入电流的改变可以引起输出电流的改变。若加一个合适的电阻R,可以实现电压的放大。这就是双极型晶体管的放大作用。
\subsubsection*{2、什么是基区扩展电阻?什么是电流聚集效应?怎么解决电流聚集效应的问题?}

基区扩展电阻是BJT有源电阻和无源电阻之和。

电流聚集效应:由于基区扩展电阻的存在,当有基极电流流过时,有源区和无源区都会产生一个横向电位降,使发射结结面上有非平衡载流子注入,由于非平衡载流子的注入,使沿着发射结有非平衡电流的分布,使发射结边缘有很高的电流密度,这就是电流聚集效应。

交叉指状电极可以消除电流聚集效应。
\subsubsection*{3、造成 BJT 频率响应的原因是信号的时间延迟,解释时间延迟现象。}

造成BJT频率响应的原因是由于信号从发射结输入到达集电结有一定的时间延迟。发射结输入信号,随着信号的变化,BJT中的电子排布也相应发生变化,使输出信号发生变化。在低频时,BJT中的电子排布能跟上输入信号的变化,使输出信号即时与输入信号变化。当频率达到一定高度时,BJT中的电子排布跟不上信号的变化,使输出信号与输入信号有一定的时间延迟。这就是时间延迟现象。
\subsubsection*{4、造成 BJT 频率响应的原因是信号的时间延迟,列举出引起信号延迟的主要的 4 个因素,并分别进行解释。写出总的信号延迟时间表达式。}

4个因素:发射结过渡电容充电时间,基区渡越时间,集电结渡越时间,集电结电容充电时间。\\
发射结电容充电时间$\tau_{E}$:由于BJT由pn结构成,有电容效应。在频率较高时,电容充放电需要一定时间。\\基区渡越时间$\tau_{B}$:载流子穿过基区薄层的时间.\\集电结过渡时间$\tau_{D}$:由于集电结的空间电荷区比较宽,载流子穿越需要一定时间。\\集电结电容充电时间$\tau_{C}$:由于集电结也是由pn结构成,因此在高频时充放电也需要时间。
$$\tau_{EC}=\tau_{E}+\tau_{B}+\tau_{D}+\tau_{C}$$
\subsubsection*{5、什么是双极结型晶体管的开关时间?开关时间包括 4 种时间,请分别说明这 4 种时间的物理意义。画出集电极电流的典型开关波形图,在图上标出这几种时间。}

当BJT工作在饱和区时,由于允许通过的电流很大,阻抗很低,常被称作开状态;当BJT工作在截至区时,允许通过的电流很小,阻抗很大,被称作关状态。由于从开状态到关状态的转变是晶体管中的载流子的分布情况的改变,而在转变时,载流子的分布并不能瞬间完成,而是需要一个过渡时间。因此由开状态转变为关状态的时间就是开关时间。\\
1.导通延迟时间:从加上输入脉冲到电流达到0.1$I_{CS}$的时间,\\2.上升时间:电流由0.1$I_{CS}$,到达0.1$I_{CS}$的时间,\\3.存储时间:从基区发生负阶跃到基区电流达到0.9$I_{CS}$的时间\\4.下降时间:电流由0.9$I_{CS}$下降到0.1$I_{CS}$的时间。
\section{金属-半导体结}
\subsection*{一、简答题}
\subsubsection*{1、 肖特基势垒具有单向导电性(即整流特性),试进行分析。}

当在半导体相对于金属一侧加一负电压V,半导体一侧的所有能级都上升qV,电势差下降至fai0-V,势垒高度下降至q(fai0-v),由于势垒高度下降,有助于电子从半导体一侧运动到金属一侧,形成较大的电流,faib基本保持不变,这种被称作正向偏压。

反的在半导体一侧相对于金属一侧加一正电压,半导体一侧的所有能级都下降qV,电势差升高至fai0+Vr,势垒高度升高至q(fai0+Vr),由于势垒高度的增加,阻碍了电子由半导体一侧向金属一侧的迁移,形成了很小的电流。faib同样基本保持不变,称为反向偏压。

以上分析便是肖特基势垒的单向导电性,即整流特性。
\subsubsection*{2、 什么是肖特基二极管的界面态(表面态)?界面态(表面态)有哪些特点?}

在实际肖特基二极管中,在界面处由于晶格的断裂而产生大量能量的现象叫做界面态或表面态。在禁带中发生。

界面态按能量连续排布,可以用中性能量$E_{0}$表示。当被占据的界面态高达$E_{0}$,且$E_{0}$上方为空时,界面态表面显中性。也就是说,当$E_{0}$下方为空时,界面态表面显正电荷,相当于施主能级的作用。当$E_{0}$上方为空时,显负电荷。相当于受主能级的作用。当$E_{0}$与$E_{f}$重叠时,表面不显电荷。

界面态具有负反馈作用,它总是有将$E_{0}$和$E_{f}$相接近的趋势。当界面态密度很高时,$E_{f}$实际上被钳位在$E_{0}$,这就是费米能级的钉扎效应,而与金属和半导体的波函数无关。
\subsubsection*{3、 什么是肖特基效应?解释肖特基效应的物理机制。}

在镜像力的作用下,肖特基势垒中电子能量降低,也就是肖特基势垒高度降低的现象称为肖特基效应。
\\物理机制:在金属表面附近x处的一个电子会在金属表面感应出一个正电荷,金属表面x处的电子与这个正电荷之间的作用力大小等同于金属表面x处的电子与-x处的等量电荷之间的作用力。这个作用力被称为镜像力,这个感应电荷被称作镜像电荷。由于镜像力会在势垒处产生一个电场,使肖特基势垒的电子能量在x=0处减小,也就是使肖特基势垒高度减小。这就是肖特基效应。

\subsubsection*{4、 什么是欧姆接触?如何形成欧姆接触?对于金属-p 型半导体,什么情况下是欧姆接触?什么情况是整流结?如果是金属-n 型,什么情况下是欧姆接触?什么情况是整流结?}

欧姆接触表示为这样一种接触:所使用的结构不会添加较大的寄生电阻,也不会使载流子的浓度发生改变以至于器件的特性发生改变。

在重参杂的情况下,金属和半导体的接触为欧姆接触。欧姆接触的物理机制:载流子隧穿肖特基势垒而不是越过肖特基势垒。

金属-P型半导体:\\
faim>fais  \ \ \ 欧姆结\\
faim<fais  \ \ \ 整流结

金属-n型半导体:\\
faim>fais  \ \ \ 整流结\\faim>fais  \ \ \ 欧姆结
\section{结型场效应管和金属半导体场效应管}
\subsection{一、简答题}
\subsubsection*{1、 什么结型场效应晶体管(即 JEFT)?解释沟道夹断、漏电流饱和、夹断电压的概念。}
Jfet是pn结型场效应晶体管,它利用pn结作为栅极来控制欧姆结两端的电阻,从而控制欧姆结两端的电流。JFET本质上是一个电压控制的电阻,它是一个单极性晶体管。

沟道夹断:随着漏极电压的增加,在x=L处,空间电荷区连接,非平衡载流子耗尽的现象称为沟道夹断。

漏电流饱和:沟道夹断时的饱和漏电流,记作$I_{ds}$

夹断电压:夹断点处的电势Vp,称为夹断电压。
\subsubsection*{2、 目前用的比较多的是 GaAs MESFET,为什么?为什么没有 Si MESFET?}

硅MESFET制作很困难,需要花大力气去防止金属淀积前半导体表面出现自然氧化的现象。而GaAs的势垒高度为0.72-0.90V,在金属淀积前,半导体表面不敏感。

更重要的是砷化镓的电子陷阱不影响电压对耗尽层宽度的调控,这为砷化镓MESFET的制备提供了很大帮助。

而且,砷化镓的电子迁移率是硅的6倍,适用于高频。
\subsubsection*{3、 什么是截止频率 fC0?截止频率 fC0由什么决定?如何实现最好的高频性能?}

随着频率的升高,电流增益的会逐渐降低,截至频率定义为:当电流增益下降为1时,即不对输入信号进行放大时的频率。

截至频率的公式为:$$f_{co}=\dfrac{V_{p0}u_{n}}{2L^{2}}$$
可见影响因素为:夹断电压、电子迁移率、沟道长度。

当频率很高时,夹断电压一般不容易改变,因此未获得更好的高频特性,需要大的电子迁移率和小的沟道长度
\subsubsection*{4、 MESFET 有两种类型,分别进行解释说明。}
常闭型或增强型:$V_{G}$=0时,MESFET的肖特基势垒穿透了n型砷化镓的外延层达到绝缘衬底,因此导电沟道不出现

常开型或耗尽型:$V_{G}$=0时,MESFET的肖特基势垒没有到达绝缘层衬底,因此导电沟道存在。
\subsubsection*{5、 什么是二维电子气(2-DEG)?}
当电子势阱的深度很深时,电子被限制在势阱所决定的薄层中,这种系统被称作二维电子气。
二维电子气是指电子(或空穴),在平行于界面的空间内可自由运动,在垂直于界面的空间内受到限制,也就是说在垂直于界面的空间,电子的动量是量子化的。由于电子的势阱深度受栅极偏压$V_{G}$的影响,二维电子气的浓度受$V_{G}$影响,因此器件的电流受栅极电压$V_{G}$影响。
\end{document}
